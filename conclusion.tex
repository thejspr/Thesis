% \meta{The conclusion should summarise the work and the final product.}
% What the thesis sai
\fixme{still a draft}

BDD works well, but can be hard with new technology. As a design flaws are
detected which might have been prevented with a more thorough design upfront.

For most applications, processor power is not an issue. Whats more important
are disk and network IO.


\section{Future Work}
% \meta{Covers which what could be improved if more time was available}
Given more time for development, there are a number of imminent features and
a bug which should be added or fixed before Yarn will be stable enough to be
used in a production environment.

The bug regarding dropped TCP connections should be fixed, possibly by
enqueuing incomming requests to make sure requests are being properly served
even at hight volumes. Implementing a request queue with the current process
concurrency model will require some form of inter-process communication
protocol.

%Support keep-alive connections to save TCP overhead.
Yarn currently does not support persistent TCP connections, and for each
request, a new TCP connection is used. This creates an overhead which could be
spared by allowing clients to reuse existing connections.

%opensource.
The source-code for Yarn is open-source and publicly available on Github which
is a web-based collaborative Git hosting site. This enables anyone to suggest
or add improvements and features to the project for the good of all Yarn
users. Given time to mature and possible interest from the Ruby community,
Yarn might evolve to be a prominent Ruby webserver.
