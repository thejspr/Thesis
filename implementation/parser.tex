\section{Parser}
% \meta{describe how the parser works and the alternatives considered}

The HTTP request is received as a string of text, and in order to know the how
to respond, it needs to be checked for it's validity. Furthermore, it should
be stored in a more convenient data format, to enable easier looking up
request information.

\subsection{HTTP Requests}
%RFC
The current version of the HTTP protocol, version 1.1, is well defined in a
specification called RFC 2616 \cite{rfc}. The specification clearly defines how
a HTTP request is formed and what it consists of. Listing~\ref{bnf} is an
excerpt from the specification and shows the overall structure of a HTTP
request.

\bigskip
\begin{lstlisting}[label=bnf,caption=HTTP request structure]
Request = Request-Line
          *(( general-header
           | request-header
           | entity-header ) CRLF)
          CRLF
          [ message-body ]
\end{lstlisting}

Listing~\ref{bnf} translates to; a \texttt{Request-Line}, followed by zero or
more headers, and after a \texttt{CRLF} an optional \texttt{message-body}.
Headers include information such as cookies present in the browser, browser
type, size of the \texttt{message-body}. \texttt{CRLF} is a newline (line
break), and is used to define the structure of the request, e.g.\ each
header is a separate line and an empty line separate the last header and the
\texttt{message-body}. The \texttt{Request-Line}, as shown in
Listing~\ref{reqline}, consists of a method, URI and HTTP version.

\bigskip
\begin{lstlisting}[label=reqline,caption=HTTP Request-Line]
Request-Line = Method SP Request-URI SP HTTP-Version CRLF
\end{lstlisting}

Listing~\ref{req_example} shows an example HTTP request.

\bigskip
\begin{lstlisting}[label=req_example,caption=Example HTTP request]
GET /path/file.html HTTP/1.1
User-Agent: Internet Explorer
Content-Length: 21

field=value&show=true
\end{lstlisting}

% different parsers
\subsection{Parsing Strategy}
The parser was written using a Ruby library named Parslet. Parslet works by
defining a set of rules and then applying them to check whether the input
satisfies the rules, and then extracts defined values. The rules consists of a
mix of regular expressions and boolean logic. Listing~\ref{parsersrc} shows the rule for the \texttt{Request-Line} as shown in Listing~\ref{reqline}.

\bigskip
\begin{lstlisting}[label=parsersrc,caption=Request-Line parser rule.
(Yarn/lib/yarn/parser.rb:61)]
rule(:request_line) do 
  method.as(:method) >> 
  space >> 
  request_uri.as(:uri) >> 
  space >> 
  http_version.as(:version) >> 
  crlf.maybe 
end
\end{lstlisting}

When a request is successfully parsed, a \texttt{Hash} containing the request
data is returned. E.g.\ running the parser on the example HTTP request from
Listing~\ref{req_example} will produce the following \texttt{Hash}:

\bigskip
\begin{lstlisting}[label=parser_out,caption=Example \texttt{Parser} output]
{ :method=>"GET"@0,
  :uri=>{:path=>"/path/file.html"@4},
  :version=>"HTTP/1.1"@20,
  :headers=>
    { "User-Agent"=>"Internet Explorer",
      "Content-Length"=>"21" },
  :body=>"field=value&show=true"@105 }
\end{lstlisting}

With the request formatted in a \texttt{Hash} it becomes very easy to
query information about it. The numbers in Listing~\ref{parser_out} preluded
by a @ denote the elements' start position in the request.
