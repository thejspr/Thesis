\section{Rack Applications}
\label{rack}
% \meta{describe how the Rack interface was implemented and what it required}
Implementing support for Rack applications consisted of two parts; loading the
Rack application and communicating requests and responses between the server
and the application.

Rack application are specified in a Ruby file which needs to be evaluated to
load the application. An example application which simply returns
an array is included in Listing~\ref{rackex}.

\bigskip
\begin{lstlisting}[label=rackex,caption=Sample Rack application. (test\_objects/config.ru:4)]
run Proc.new { |env|
  [200, {'Content-Type' => "text/plain"}, ["Rack works"]]
}
\end{lstlisting}

When a Rack application file is supplied when running Yarn, the contents of
the file is read, passed into the \texttt{Rack::Builder} constructor which is
then evaluated. Listing~\ref{loadrack} contains the line where the Rack is
loaded.

\bigskip
\begin{lstlisting}[label=loadrack,caption=Loading a Rack application.
(lib/yarn/server.rb:31)]
  config_file = File.read(app_path)
  rack_application = eval("Rack::Builder.new { #{config_file} }")
\end{lstlisting}

Once loaded a Rack application is invoked with a \texttt{call} method which
returns an array containing the HTTP status code, HTTP headers and message
body. The \texttt{call} method takes one parameter which is a
\texttt{Hash} of values the Rack application needs such as the path, hostname,
port, request body etc. \cite{rackspec}. 




