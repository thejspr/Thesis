% \meta{A small summary of the whole project from start to finish.} 1/2 - 3/4
% page.
\begin{abstract}
In this project, a concurrent webserver is implemented in Ruby using the
Behaviour-Driven Development (BDD) method. 

Topics relevant to the thesis, namely webservers, concurrent programming, the
Ruby programming language and BDD are covered. The implementation of each
requirement is described, along with selected technical problems and their
solutions.

Yarn, the name of the produced webserver, is benchmarked to other Ruby
webservers and the results revealed that for static requests it performed
sub par other webservers, but for CPU intensive applications it performed quite
well.

Using BDD enabled getting started on the implementation right away, without
having to produce a big design upfront. This proved to be both an advantage and
a disadvantage, as some design decisions which turned out not to work as
expected, might have been avoided. As BDD employs a test-first approach, a
large test-suite was produced and having many tests enabled for design changes
late in the development, without having to worry about the correctness of the
implementation.
\end{abstract}

