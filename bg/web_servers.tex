\section{Webservers}
\label{webservers}
\meta{Cover several webservers, each with their unique way of handling
  concurrent requests.}

The goal of a webserver is listen on a port for incoming requests, handle them
by either returning the contents of a file, run a program and return it's
output, or, if an error occurred, return an error message. Since the Internet was
made public over 20 years ago there have been countless webservers created, most
of which focus on handling websites written in a specific programming language
or framework. Multi-core processors have long been available in server
environments and hence, many webservers employ techniques to raise their
performance by handling http requests in parallel.

\fixme{Mention underlying protocols, HTTP and TCP}

This section looks at some of the current webservers and how they went about
solving the problem of handling multiple http request in parallel.

\subsection{Apache HTTP Server}
Apache HTTP Server, or simply Apache, was initially released in 1995 and is the
most used web server today surpassing 100 million websites in 2009
\cite{apachestats}. Apache supports many different functionalities and
programming languages as it can be extended by modules. Popular modules provide
authentication, encryption, compression and multi-processing. To handle
concurrent requests Apaches main process (the parent) forks child processes. 
When using the multi-processing module, each each child process contains a fixed 
number of server threads and one listener thread which divides incoming request
amongst the server threads. This increases the throughput of Apache
significantly when dealing with a high volume of concurrent requests
\cite{apachempm}.

\subsection{Rack}
Rack is not an actual webserver, but an interface between webservers supporting
Ruby and Ruby frameworks. It is covered here as the Ruby webservers covered next
both implement it. Rack decouples the link between the webserver and
web-application (written in one of the many Ruby frameworks) and hereby makes it
possible to change webserver without having to make any changes to the web
application. This is achieved by wrapping requests and responses in a Ruby
object. By implementing the Rack interface, a webserver would immediately become
compatible with more than 15 Ruby frameworks \cite{rackspec}.

\subsection{Thin}
Thin is a webserver written in Ruby which include a http parser written in C,
implements the Rack interface and uses EventMachine to handle requests in a
concurrent non-blocking manner. EventMachine is a library which enables
event-driven (non-blocking) I/O by utilizing the reactor design pattern. The
reactor pattern works by having a dispatcher (reactor) passing on work to
various handlers, which then in turn fire an event when finished. This enables
Thin to focus on other requests instead of waiting on blocking actions such as
waiting for a file to be written to disk or fetching data from a remote API
\cite{reactor}.

% \subsection{Unicorn} mongrel?

