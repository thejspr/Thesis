\section{Webservers}
\label{webservers}
% \meta{Cover several webservers, each with their unique way of handling
%   concurrent requests.}

The goal of a webserver is listen on a port for incoming requests, handle them
by either returning the contents of a file, run a program and return it's
output, or, if an error occurred, return an error message. Since the Internet was
made public over 20 years ago there have been countless webservers created, most
of which focus on handling websites written in a specific programming language
or framework. Multi-core processors have long been available in server
environments and hence, many webservers employ techniques to raise their
performance by handling HTTP requests in parallel.

This section looks at some of the available Ruby webservers and the
concurrency models they employ.

\subsection{Rack}
Rack is not an actual webserver, but an interface between webservers
supporting Ruby and Ruby frameworks. It is covered here as some of the Ruby
webservers covered next both implement it. Rack decouples the link between the
webserver and web-application (written in one of the many Ruby frameworks) and
hereby makes it possible to change webserver without having to make any
changes to the web application. This is achieved by wrapping requests and
responses in a Ruby object. By implementing the Rack interface, a webserver
would immediately become compatible with more than 15 Ruby frameworks
\cite{rackspec}.

\subsection{Thin}
Thin is a webserver written in Ruby which include a HTTP parser written in C,
implements the Rack interface and uses EventMachine to handle requests in a
concurrent non-blocking manner. EventMachine is a library which enables
event-driven (non-blocking) I/O by utilizing the reactor design pattern. The
reactor pattern works by having a dispatcher (reactor) passing on work to
various handlers, which then in turn fire an event when finished. This enables
Thin to focus on other requests instead of waiting on blocking actions such as
waiting for a file to be written to disk or fetching data from a remote API
\cite{reactor}.

Thin works by running the EventMachine event loop in a single thread, and
does therefore not employ any parallel processing, but does allow for
concurrency due to the asynchronous nature of the reactor pattern. Thin is
usually setup as a cluster of processes behind a proxy server or load
balancer.

\subsection{WEBrick}
WEBrick is a webserver which is included in the Ruby standard library. It is
written entirely in Ruby and employs concurrency by creating a separate thread
for each incoming request. Like Thin, multiple instances of WEBrick would be
required to enable multiprocessing. WEBrick is mostly used as a development
server as it does not perform well under heavy loads due to it's concurrency
model.

\subsection{Unicorn}
\fixme{Write about the Unicorn webserver if used in tests}
