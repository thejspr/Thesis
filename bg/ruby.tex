\section{Ruby} % (fold)
\label{sec:ruby}

%ruby in general
Ruby is an dynamic, reflective object-oriented general-purpose programming
language. It is dynamic in the sense that it is interpreted at runtime, and
reflective in the sense that it can modify and inspect program behavior during
runtime.  Ruby has a dynamic type system where the type of the object is
determined by what the object can do in terms of which methods are available
for the object. This type system is referred to as duck typing, i.e.\ if it
walks like a duck and talks like a duck, then the interpreter will treat it
like a duck.

The latest version of Ruby, version 1.9, holds several improvements over the
older but still maintained version 1.8. Besides small syntactic changes and
including a new VM, Ruby 1.9 introduced the concept of fibers. Ruby 1.9 also
includes RubyGems which enables easy packaging, installation and distribution
of Ruby software. RubyGems, mostly referred to as \textit{gems}, can be used
to modify or extend functionality within a Ruby application or to split out
reusable code for others to benefit from \cite{rubygems}. There are currently
over 27.000 gems available and this greatly reduces the need to "reinvent the
wheel", hence improving productivity. 

Ruby was chosen for this project because it is dynamic, expressive, productive
and terse, all adding to the speed and joyfulness of developing software.
Furthermore, Ruby has several constructs available for concurrent programming
which are covered below.

\subsection{Concurrency in Ruby}
Ruby comes with several constructs for concurrent programming; processes, threads
and fibers. The following describes the advantages and possible disadvantages
of each of these concepts.

\subsubsection{Processes}
A process is a running instance of a program, e.g.\ a running webserver or an
open text editor. Processes are handled by the operating system (OS) which
will utilize the available processors to run processes in parallel.
Hence, parallelism is achieved by running multiple instances (processes) of
the same program.

Ruby includes a library for creating, killing and inspecting processes.
Creating a new process is done by \textit{forking} a new process from the
current one. Forking starts a new Ruby VM with the executing program as a new
process. The forked and the original process does not have access to each
other memory, and thereby eliminates many of the problems regarding
race-conditions and deadlocks. The disadvantage of using multiple processes is
that, as they cannot share state, they have to communicate through an explicit
protocol \cite{ruby19}.

\subsubsection{Threads}
A thread is an encapsulation of a set of instructions and it's execution can
be managed. A thread can be seen as a lightweight process which has access to
the current programs memory. Threads are managed by a thread-scheduler which
decides when threads are executed and preempted to enable other threads to
run. Using threads introduces complex issues as race-conditions can easily
occur due to all threads having access to the same memory.

The advantages of using threads is that they can easily share data, but that
is also one of it's disadvantages as it can easily become quite complex and
bugs concerning multi-threaded code can be very hard to discover \cite{ruby19}.

\subsubsection{Fibers}
A fiber in Ruby is a coroutine mechanism which enables pausing and resuming
code blocks to achieve cooperative concurrency. A code block is a
piece of ruby code encapsulated in an object, and can be passed into
methods, saved in variables, and executed on demand. Fibers resembles threads in
that they can be controlled from outside, but whereas threads are managed by a
scheduler, fibers must be scheduled by the programmer.

The advantages of fibers over threads are that they have a significantly lower
memory footprint and doesn't introduce the complexity of a scheduler as
threads do \cite{rubyfiber}.

\subsection{Ruby Implementations}
Ruby code is run in a virtual machine (VM) of which there exists several
implementations. The official Ruby VM, for Ruby 1.9, is called YARV (Yet
Another Ruby VM). As there is no specification for Ruby, YARV serves as
the reference implementation for the Ruby language. The following describes
two widely used Ruby implementations, namely YARV and JRuby, and how they
differ regarding concurrent programming.

\subsubsection{Yet Another Ruby VM}
YARV is developed and maintained as the official Ruby implementation and replaced
the old VM MRI (Matz\footnote{Yukihiro Matsumoto, a.k.a.\ Matz is the creator of 
the Ruby programming language} Ruby Interpreter) as of Ruby version 1.9. Given
YARV is the one driving the Ruby language implementation, it always has the
latest languages features and updates, whereas other Ruby implementations tend
to be a bit behind. The latests stable version of YARV is 1.9.2, with 1.9.3
being just around the corner. YARV supports threads, fibers and processes, but
includes a Global Interpreter Lock (GIL) which limits the VM to only running
one thread at a time. The GIL is included to ensure non-threadsafe
C-extensions does not cause problems, and effectively bars YARV from
utilizing true parallelism from fibers and threads \cite{ruby19}.

\subsubsection{JRuby}
JRuby is a Ruby implementation that runs on any Java Virtual Machine (JVM)
and therefore enables using Java libraries from within Ruby. Java is a very
mature and large ecosystem with many useful libraries, language constructs and
concurrency models \cite{usejruby}.

JRuby does not include a GIL as YARV does and therefore allows true
parallelism for threads, as opposed to only having one thread/fiber run
at a time. JRuby does however not support processes forking as it is not
supported by the JVM\@ \cite{jruby}.
